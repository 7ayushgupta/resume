\cvsection{Projects}

\begin{cventries}

  \cventry
  {Inter IIT Tech Meet}
  {\href{https://github.com/yashsriv/beethoven}{Dashboard Web Extension/App}}
  {\emph{\texttt{\href{https://github.com/yashsriv/beethoven}{github://yashsriv/beethoven}}}}
  {February, 2017}
  {
    \begin{cvitems}
    \item Involved creating a web extension which acted as a user’s home page
      \ifdefined \ONEPAGE and displayed relevant information to a student. \else
      and helped display all the information relevant to a student studying in
      IIT Kanpur in a main Dashboard. (News, Events, Student Search, Share Auto)
      \fi
    \item Server Side was implemented as multiple \textbf{microservices} in
      various languages \ifdefined \ONEPAGE \else (Golang, Python, NodeJS) \fi
      involving IPC via \textbf{JSON RPC}.
    \item Fully \textbf{Dockerized backend} running on a docker-compose setup.
      Written with \textbf{scalability} and speed in mind.
    \item Judged \textbf{1\textsuperscript{st}} among all IITs participating
      in the competition.
    \end{cvitems}
  }

  \cventry
  {Member, Team Robocon IIT Kanpur\ifdefined\ONEPAGE\else, Prof. Bhaskardas Gupta\fi}
  {ABU Robocon 2016}
  {IIT Kanpur}
  {Oct'2015 - Mar'2016}
  {
    \begin{cvitems}
      \item An autonomous robot, which did not contain a driving actuator had to
        traverse a game field using energy provided to it by another robot in
        form of a non contact force.
      \item Was involved in \textbf{Image Processing} used in the autonomous
        robot for \textbf{color detection} and \textbf{line following}
        \ifdefined \ONEPAGE \else to traverse the arena. \fi
      \item Came \textbf{3\textsuperscript{rd}} out of 105 teams in Nationals at Pune, India.
    \end{cvitems}
  }

  \smallcventry
  {Programming Club}
  {\href{http://pclub.in/project/2016/07/06/smartmirror.html}{Smart Mirror}}
  {Best Applicable Project}
  {Summer'2016}
  {\emph{\texttt{\href{https://github.com/11000011/Smart-Mirror}{github://11000011/Smart-Mirror}}}}
  {
    \begin{cvitems}
    \item Built an \textbf{IoT Mirror} with an RPi and a display fitted with a 75\%
      reflecting mirror.
    \item The mirror had features such as weather forecast, calendar
      and pushbullet notifications of a user (determined via face
      identification).
      \item I was involved in integrating Google Calendar notifications and Face
        Identification using Microsoft’s Project Oxford API.
    \item Received \textbf{Best Applicable Project} amongst all summer projects under
      the Science and Technology Council, IIT Kanpur.
    \end{cvitems}
  }

  \smallcventry
  {Association of Computing Activities}
  {\href{https://github.com/yashsriv/Reversi-Python}{Reversi Game in Python}}
  {IIT Kanpur}
  {2\textsuperscript{nd} Semester}
  {\emph{\texttt{\href{https://github.com/yashsriv/Reversi-Python}{github://yashsriv/Reversi-Python}}}}
  {
    \begin{cvitems}
    \item Developed a Python Application using \textbf{Pygame} for 2 player as well as
      single player Reversi gameplay.
    \item Uses the \textbf{negamax algorithm} with an efficient heuristic check
      for better performance against humans.
    \item Mid Semester project under the Association of Computing Activities (ACA), IIT Kanpur.
    \item Link: \href{https://github.com/yashsriv/Reversi-Python}{github.com/yashsriv/Reversi-Python}
    \end{cvitems}
  }

  % \smallcventry
  % {Self Project}
  % {\href{https://github.com/yashsriv/go-nachos}{go-nachos}}
  % {Operating Systems}
  % {Dec'2017}
  % {}
  % {A port of the educational OS, nachos, in golang}

  \smallcventry
  {Course Project, Compiler Design}
  {\href{https://github.com/yashsriv/tango}{tango} \strong{(\emph{golang to x86 assembly})} }
  {}
  {Jan'2018-April'2018}
  {\emph{\texttt{\href{https://github.com/yashsriv/tango}{github://yashsriv/tango}}}}
  {
    \begin{cvitems}
    \item A compiler for go written in go in a team of 3. Compiles from golang
      to x86 assembly.
    \item Supports a subset of the go language including nested pointers, type
      checking, recursion, nested arrays, structs, methods and other common
      programming language features.
    \item Added a new for comprehension syntax as well to golang.
    \end{cvitems}
  }

  \smallcventry
  {Course Project, Computer Architecture}
  {\href{https://github.com/yashsriv/branch-predictor/blob/master/report/main.pdf}{Branch Predictor}}
  {Best Predictor}
  {April'2018}
  {\emph{\texttt{\href{https://github.com/yashsriv/branch-predictor/blob/master/report/main.pdf}{github://yashsriv/branch-predictor}}}}
  {
    \begin{cvitems}
    \item Designed a branch predictor for an intra-class branch prediction
      championship based on the CBP-1 framework in a team of 2.
    \item Created a modified GEHL predictor with an additional loop predictor.
    \item Was adjudged the \textbf{best predictor} amongst all submitted.
    \end{cvitems}
  }

  \smallcventry
  {Course Project}
  {\href{https://github.com/yashsriv/haskell-connect-4}{Connect 4 AI in haskell}}
  {Functional Programming}
  {Jan'2018-April'2018}
  {\emph{\texttt{\href{https://github.com/yashsriv/haskell-connect-4}{github://yashsriv/haskell-connect-4}}}}
  {
    \begin{cvitems}
    \item A GUI based connect 4 AI in haskell.
    \item Had support for various difficulties and the AI was abstracted out in
      order to be able to support any complete knowledge two player game.
    \end{cvitems}
  }

  \smallcventry
  {24 Hour Hackathon}
  {Code.Fun.Do}
  {Microsoft India}
  {Sept'2015}
  {Best 5 ideas}
  {
    \begin{cvitems}
    \item Developed an App to help connect teachers and learners based on their
      preference of subjects.
    \item Used cross-platform \textbf{Universal App Platform} for Windows 10
      and a server written in C\#.
    \item Was selected as one of the \textbf{best five ideas}.
    \end{cvitems}
  }

  \smallcventry
  {Self Project}
  {\href{https://github.com/yashsriv/go-nachos}{go-nachos}}
  {Ported nachOS to golang}
  {Dec'2017}
  {\emph{\texttt{\href{https://github.com/yashsriv/go-nachos}{github://yashsriv/go-nachos}}}}
  {}

\end{cventries}
\vspace{-2mm}

%%% Local Variables:
%%% mode: latex
%%% TeX-master: "../cv.tex"
%%% TeX-engine: xelatex
%%% End: