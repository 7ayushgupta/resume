%!TEX TS-program = xelatex
%!TEX encoding = UTF-8 Unicode
% Awesome CV LaTeX Template for CV/Resume
%
% This template has been downloaded from:
% https://github.com/posquit0/Awesome-CV
%
% Author:
% Claud D. Park <posquit0.bj@gmail.com>
% http://www.posquit0.com
%
% Template license:
% CC BY-SA 4.0 (https://creativecommons.org/licenses/by-sa/4.0/)
%


\documentclass[10pt, a4paper]{awesome-cv}
\geometry{left=1.4cm, top=.8cm, right=1.4cm, bottom=1.8cm, footskip=.5cm}
\fontdir[fonts/]

% Color for highlights
% Awesome Colors: awesome-emerald, awesome-skyblue, awesome-red, awesome-pink, awesome-orange
%                 awesome-nephritis, awesome-concrete, awesome-darknight
\colorlet{awesome}{awesome-darknight}
% Uncomment if you would like to specify your own color
% \definecolor{awesome}{HTML}{CA63A8}

% Colors for text
% Uncomment if you would like to specify your own color
% \definecolor{darktext}{HTML}{414141}
% \definecolor{text}{HTML}{333333}
% \definecolor{graytext}{HTML}{5D5D5D}
% \definecolor{lighttext}{HTML}{999999}

% Set false if you don't want to highlight section with awesome color
\setbool{acvSectionColorHighlight}{false}

% If you would like to change the social information separator from a pipe (|) to something else
\renewcommand{\acvHeaderSocialSep}{\quad\textbar\quad}


% Available options: circle|rectangle,edge/noedge,left/right
% \photo{./profile.png}
\name{Yash}{Srivastav}
\position{Senior Undergraduate{\enskip\cdotp\enskip}Computer Science and Engineering}
\address{Indian Institute of Technology, Kanpur}
\mobile{(+91) 705-413-3662}
\email{yash111998@gmail.com}
\homepage{yashsriv.org}
\github{yashsriv}
\linkedin{yashsriv}
% \twitter{@therealyashsriv}
% \quote{``There is no fate but what we make."}

\newcommand{\smallcventry}[6]{\cventry{#1}{#2}{#3}{#4}{#6}}
\newcommand{\specialcvsection}[1]{\cvsection{#1}}




\begin{document}
\makecvheader
\makecvfooter
  {}
  {}
  {\thepage}

\cvsection{Education}

\begin{cventries}

  \cventry
    {Bachelor Of Technology, Mechanical Engineering}
    {Indian Institute of Technology Kanpur}
    {Kanpur, India}
    {July 2017 - Present}
    {
      \begin{cvitems}
        \item {Cumulative Grade Point: 9.2/10.0}
      \end{cvitems}
    }

  \cventry
    {Indian School Certificate Examination (Intermediate)}
    {City Montessori School}
    {Lucknow, India}
    {March 2017}
    {
      \begin{cvitems}
        \item {Overall Percentage: 95.5\%}
      \end{cvitems}
    }

  \cventry
    {Indian Certificate Of Secondary Education (High School)}
    {St. Francis' College}
    {Lucknow, India}
    {March 2015}
    {
      \begin{cvitems}
        \item {Overall Percentage: 95\%}
      \end{cvitems}
    }


  
\end{cventries}


\cvsection{Honors and Achievements}
\begin{cvhonors}

  \cvhonor
    {2nd in 15+ teams}
    {Student AUV Competition (SAVe), \break organised by\textbf{ NIOT, Chennai} in 2019} 
    {Chennai} 
    {2019}
 
  \cvhonor
  {Top 0.7\%}
  {JEE Advanced (amongst 160,000 candidates)}
  {}
  {2017}

  \cvhonor
  {Top 0.1\%}
  {JEE Main (amongst 1.3 million candidates)}
  {}
  {2017}
  
  \cvhonor
  {Top 1\% (U.P)}
  {National Standard Examination in \textbf{Physics}, 2016, appeared for \textvf{INPho} 2017}
  {India}
  {2016}
 
  \cvhonor
  {Top 1\% (India)} 
  {National Standard Examination in \textbf{Chemistry}, 2016, appeared for \textvf{INCho} 2017}
  {India}
  {}

\end{cvhonors}

%%% Local Variables:
%%% mode: latex
%%% TeX-engine: xetex
%%% TeX-master: "../cv"
%%% End:
%-------------------------------------------------------------------------------
%	SECTION TITLE
%-------------------------------------------------------------------------------
\cvsection{Experience}


%-------------------------------------------------------------------------------
%	CONTENT
%-------------------------------------------------------------------------------
\begin{cventries}

%---------------------------------------------------------
  \cventry
    {Software Architect} % Job title
    {Omnious. Co., Ltd.} % Organization
    {Seoul, S.Korea} % Location
    {Jun. 2017 - May. 2018} % Date(s)
    {
      \begin{cvitems} % Description(s) of tasks/responsibilities
        \item {Provisioned an easily managable hybrid infrastructure(Amazon AWS + On-premise) utilizing IaC(Infrastructure as Code) tools like Ansible, Packer and Terraform.}
        \item {Built fully automated CI/CD pipelines on CircleCI for containerized applications using Docker, AWS ECR and Rancher.}
        \item {Designed an overall service architecture and pipelines of the Machine Learning based Fashion Tagging API SaaS product with the micro-services architecture.}
        \item {Implemented several API microservices in Node.js Koa and in the serverless AWS Lambda functions.}
        \item {Deployed a centralized logging environment(ELK, Filebeat, CloudWatch, S3) which gather log data from docker containers and AWS resources.}
        \item {Deployed a centralized monitoring environment(Grafana, InfluxDB, CollectD) which gather system metrics as well as docker run-time metrics.}
      \end{cvitems}
    }

%---------------------------------------------------------
  \cventry
    {Co-founder \& Software Engineer} % Job title
    {PLAT Corp.} % Organization
    {Seoul, S.Korea} % Location
    {Jan. 2016 - Jun. 2017} % Date(s)
    {
      \begin{cvitems} % Description(s) of tasks/responsibilities
        \item {Implemented RESTful API server for car rental booking application(CARPLAT in Google Play).}
        \item {Built and deployed overall service infrastructure utilizing Docker container, CircleCI, and several AWS stack(Including EC2, ECS, Route 53, S3, CloudFront, RDS, ElastiCache, IAM), focusing on high-availability, fault tolerance, and auto-scaling.}
        \item {Developed an easy-to-use Payment module which connects to major PG(Payment Gateway) companies in Korea.}
      \end{cvitems}
    }

%---------------------------------------------------------
  \cventry
    {Researcher} % Job title
    {Undergraduate Research, Machine Learning Lab(Prof. Seungjin Choi)} % Organization
    {Pohang, S.Korea} % Location
    {Mar. 2016 - Exp. Jun. 2017} % Date(s)
    {
      \begin{cvitems} % Description(s) of tasks/responsibilities
        \item {Researched classification algorithms(SVM, CNN) to improve accuracy of human exercise recognition with wearable device.}
        \item {Developed two TIZEN applications to collect sample data set and to recognize user exercise on SAMSUNG Gear S.}
      \end{cvitems}
    }

%---------------------------------------------------------
  \cventry
    {Software Engineer \& Security Researcher (Compulsory Military Service)} % Job title
    {R.O.K Cyber Command, MND} % Organization
    {Seoul, S.Korea} % Location
    {Aug. 2014 - Apr. 2016} % Date(s)
    {
      \begin{cvitems} % Description(s) of tasks/responsibilities
        \item {Lead engineer on agent-less backtracking system that can discover client device's fingerprint(including public and private IP) independently of the Proxy, VPN and NAT.}
        \item {Implemented a distributed web stress test tool with high anonymity.}
        \item {Implemented a military cooperation system which is web based real time messenger in Scala on Lift.}
      \end{cvitems}
    }

%---------------------------------------------------------
  \cventry
    {Game Developer Intern at Global Internship Program} % Job title
    {NEXON} % Organization
    {Seoul, S.Korea \& LA, U.S.A} % Location
    {Jan. 2013 - Feb. 2013} % Date(s)
    {
      \begin{cvitems} % Description(s) of tasks/responsibilities
        \item {Developed in Cocos2d-x an action puzzle game(Dragon Buster) targeting U.S. market.}
        \item {Implemented API server which is communicating with game client and In-App Store, along with two other team members who wrote the game logic and designed game graphics.}
        \item {Won the 2nd prize in final evaluation.}
      \end{cvitems}
    }

%---------------------------------------------------------
  \cventry
    {Researcher for <Detecting video’s torrents using image similarity algorithms>} % Job title
    {Undergraduate Research, Computer Vision Lab(Prof. Bohyung Han)} % Organization
    {Pohang, S.Korea} % Location
    {Sep. 2012 - Feb. 2013} % Date(s)
    {
      \begin{cvitems} % Description(s) of tasks/responsibilities
        \item {Researched means of retrieving a corresponding video based on image contents using image similarity algorithm.}
        \item {Implemented prototype that users can obtain torrent magnet links of corresponding video relevant to an image on web site.}
      \end{cvitems}
    }

%---------------------------------------------------------
  \cventry
    {Software Engineer Trainee} % Job title
    {Software Maestro (funded by Korea Ministry of Knowledge and Economy)} % Organization
    {Seoul, S.Korea} % Location
    {Jul. 2012 - Jun. 2013} % Date(s)
    {
      \begin{cvitems} % Description(s) of tasks/responsibilities
        \item {Performed research memory management strategies of OS and implemented in Python an interactive simulator for Linux kernel memory management.}
      \end{cvitems}
    }

%---------------------------------------------------------
  \cventry
    {Software Engineer} % Job title
    {ShitOne Corp.} % Organization
    {Seoul, S.Korea} % Location
    {Dec. 2011 - Feb. 2012} % Date(s)
    {
      \begin{cvitems} % Description(s) of tasks/responsibilities
        \item {Developed a proxy drive smartphone application which connects proxy driver and customer. Implemented overall Android application logic and wrote API server for community service, along with lead engineer who designed bidding protocol on raw socket and implemented API server for bidding.}
      \end{cvitems}
    }

%---------------------------------------------------------
  \cventry
    {Freelance Penetration Tester} % Job title
    {SAMSUNG Electronics} % Organization
    {S.Korea} % Location
    {Sep. 2013, Mar. 2011 - Oct. 2011} % Date(s)
    {
      \begin{cvitems} % Description(s) of tasks/responsibilities
        \item {Conducted penetration testing on SAMSUNG KNOX, which is solution for enterprise mobile security.}
        \item {Conducted penetration testing on SAMSUNG Smart TV.}
      \end{cvitems}
      %\begin{cvsubentries}
      %  \cvsubentry{}{KNOX(Solution for Enterprise Mobile Security) Penetration Testing}{Sep. 2013}{}
      %  \cvsubentry{}{Smart TV Penetration Testing}{Mar. 2011 - Oct. 2011}{}
      %\end{cvsubentries}
    }

%---------------------------------------------------------
\end{cventries}

\cvsection{Skills}
\ifdefined\ONEPAGE
\\
\textbf{Robotics}: ROS, OpenCV, Arduino, Gazebo, CUDA, Gym\\
\textbf{Design}: Solidworks 2018, AutoCAD, Inventor, LabVIEW\\
\textbf{Data Science}: Tensorflow, Keras, Scikit, MATLAB\\
\textbf{Programming Languages}: C++, Python, Scala, Javascript\\
\else
\fi
%%% Local Variables:
%%% mode: latex
%%% End:
\cvsection{Relevant Coursework}

\begin{tabular*}{\textwidth}{l l l l}
  Engineering Design and Graphics (A$*$)& Dynamics (A) & Fluid Mechanics (A)  & Probability \& Statistics\\
  Fundamentals of Computing (A$*$) & Mechanics Of Solids (A) & Thermodynamics (A) & Complex Analysis\\
  Theory of Mechanisms and Machines ($i$) & Multi-Variable Calculus & Introduction to Robotics ($i$) & Energy Systems ($i$)

\end{tabular*}
{\footnotesize
    {A$*$: Grade for exceptional performance, $i$: In progress, A: grade}
}
%%% Local Variables:
%%% mode: latex
%%% TeX-engine: xetex
%%% TeX-master: "../cv"
%%% End:
\newpage
\cvsection{Projects}

\begin{cventries}

 \cventry
    {Faculty Advisor: Prof. Mangal Kothari}
    {Team AUV-IITK}
    {Software Team Member}
    {May 2018 - Present}
    {
      \begin{cvitems}
        \item{Designed a \textbf{hierarichal finite state machine} for robust autonomous behavior of the vehicle with failsafes}
        \item{Fused sensor readings from Doppler Velocity Log (DVL) and IMU using an \textbf{EKF} to estimate odometry}
        %\item{Implemented a novel image preprocessing algorithm based on Fusion Framework to formulate a robust underwater computer vision pipeline}
        
        \item{Developed and tested acoustic localization system capable of estimating the Direction of Arrival of ultrasonic underwater signals from pinger, using \textbf{STFT} and \textbf{Cross-Correlation}}
        % \item{Used signal processing operations such as Short Time Fourier Transform and Cross-Correlation to find time delay of arrival between signals}
        % \item{Managed a multi-layered software stack for an autonomous underwater vehicle, Anahita developed on ROS and simulated using Gazebo}
        \item{Tuned and tested Cascaded PID Controller on the vehicle, enabling it to perform waypoint navigation \& visual servoing}
        \item{Extensively used \textbf{Gazebo, a physics engine} to simulate vehicle model in a hydrodynamically realistic environment}
        % \item{Created setups for disparity map generation using a pair of cameras and implemented a modified Fast-SLAM for underwater localization}
      \end{cvitems}
    }
  
   \cventry
    {Mentor: Prof. Mangal Kothari}
    {Realtime Onboard Dense RGB-D Mapping on UAVs}
    {}
    {May 2019 - Present}
    {
      \begin{cvitems}
        \item {Studied and experimented various techniques related to 3D mapping of environment using monocular and stereo cameras on Jetson TX2 for onboard implementation}
        \item {Evaluated approaches for shortcomings and processing requirements while focussing on the scarce size, computation and energy resources on Unmanned Aerial Vehicles (UAVs)}
      \end{cvitems}
    }

  \cventry
    {Mentor: Prof. Puroshottam Kar}
    {Chat-IITK}
    {Advanced Track Project - ESC101}
    {2nd Semester}
    {
      \begin{cvitems}
        \item {Designed and developed a chat application on NodeJS, Express, and MongoDB, selected in \textbf{12} out of 400+ students}
        \item {Implemented real-time chat using Socket-IO with PassportJS for extensively implemented \textbf{authentication} and \textbf{cookie handling} for session management}
        \item {\textbf{Database management} implemented using MongoDB, and application deployed online on Heroku's server}
      \end{cvitems}
    }

    \cventry
    {Mentor: Prof. Shantanu Bhattacharya}
    {Mechanical Quadruped}
    {Course Project -TA202}
    {4th Semester}
    {
      \begin{cvitems}
        \item Designed and simulated a four-legged assembly that uses Jansen's linkage mechanism to walk using \textbf{Solidworks}
        \item Made a working model of the same under constraints of size and materials using manufacturing processes such as lathing, milling and drilling
      \end{cvitems}
    } 

  % \smallcventry
  %   {Robotics Club}
  %   {Team Humanoid, IITK}
  %   {IIT Kanpur}
  %   {Dec 2017 - April 2018} 
  %   {Software Team Member}
  %   {
  %     \begin{cvitems} 
  %       \item {Worked on a Bipedal Prototype of the humanoid bot, capable of performing statically stable walking}
  %       \item {Implemented the MATLAB simulated \textbf{inverse kinematics walking algorithm} based on ZMP criteria on the actual robot using a Robot Operating System framework}
  %       \item {Developed a Web Graphical User Interface for monitoring current status and easier debugging of servos using ROS Web Bridge Server and JavaScript, with a CSS frontend}

  %     \end{cvitems}
  %   }


\end{cventries}
\vspace{-2mm}

%%% Local Variables:
%%% mode: latex
%%% TeX-master: "../cv.tex"
%%% TeX-engine: xelatex
%%% End:


% \smallcventry
  % {Self Project}
  % {\href{https://github.com/yashsriv/go-nachos}{go-nachos}}
  % {Operating Systems}
  % {Dec'2017}
  % {}
  % {A port of the educational OS, nachos, in golang}

  % \smallcventry
  % {Course Project, Compiler Design}
  % {\href{https://github.com/yashsriv/tango}{tango} \strong{(\emph{golang to x86 assembly})} }
  % {}
  % {Jan'2018-April'2018}
  % {\emph{\texttt{\href{https://github.com/yashsriv/tango}{github://yashsriv/tango}}}}
  % {
  %   \begin{cvitems}
  %   \item A compiler for go written in go in a team of 3. Compiles from golang
  %     to x86 assembly.
  %   \item Supports a subset of the go language including nested pointers, type
  %     checking, recursion, nested arrays, structs, methods and other common
  %     programming language features.
  %   \item Added a new for comprehension syntax as well to golang.
  %   \end{cvitems}
  % }

  % \smallcventry
  % {Course Project, Computer Architecture}
  % {\href{https://github.com/yashsriv/branch-predictor/blob/master/report/main.pdf}{Branch Predictor}}
  % {Best Predictor}
  % {April'2018}
  % {\emph{\texttt{\href{https://github.com/yashsriv/branch-predictor/blob/master/report/main.pdf}{github://yashsriv/branch-predictor}}}}
  % {
  %   \begin{cvitems}
  %   \item Designed a branch predictor for an intra-class branch prediction
  %     championship based on the CBP-1 framework in a team of 2.
  %   \item Created a modified GEHL predictor with an additional loop predictor.
  %   \item Was adjudged the \textbf{best predictor} amongst all submitted.
  %   \end{cvitems}
  % }

  % \smallcventry
  % {Course Project}
  % {\href{https://github.com/yashsriv/haskell-connect-4}{Connect 4 AI in haskell}}
  % {Functional Programming}
  % {Jan'2018-April'2018}
  % {\emph{\texttt{\href{https://github.com/yashsriv/haskell-connect-4}{github://yashsriv/haskell-connect-4}}}}
  % {
  %   \begin{cvitems}
  %   \item A GUI based connect 4 AI in haskell.
  %   \item Had support for various difficulties and the AI was abstracted out in
  %     order to be able to support any complete knowledge two player game.
  %   \end{cvitems}
  % }

  % \smallcventry
  % {24 Hour Hackathon}
  % {Code.Fun.Do}
  % {Microsoft India}
  % {Sept'2015}
  % {Best 5 ideas}
  % {
  %   \begin{cvitems}
  %   \item Developed an App to help connect teachers and learners based on their
  %     preference of subjects.
  %   \item Used cross-platform \textbf{Universal App Platform} for Windows 10
  %     and a server written in C\#.
  %   \item Was selected as one of the \textbf{best five ideas}.
  %   \end{cvitems}
  % }

  % \smallcventry
  % {Self Project}
  % {\href{https://github.com/yashsriv/go-nachos}{go-nachos}}
  % {Ported nachOS to golang}
  % {Dec'2017}
  % {\emph{\texttt{\href{https://github.com/yashsriv/go-nachos}{github://yashsriv/go-nachos}}}}
  % {}
\cvsection{Positions of Responsibility}

\begin{itemize}
\item \textbf{Software Team Lead}, \emph{Team AUV-IITK}
  : \\
  Maintaining entire stack of an Autonomous Vehicle, deployed on Git, implemented using ROS, OpenCV and simulation integrated using Gazebo.
\item \textbf{Secretary}, \emph{Robotics Club, IIT Kanpur 2018-19}
\item \textbf{Secretary}, \emph{Consulting Hobby Group, IIT Kanpur 2018-19}
\item \textbf{Student Guide}, \emph{Counselling Service, 2018-19}
\item \textbf{Academic Mentor}, \emph{Counselling Service, 2018-19}
\end{itemize}

  % \begin{cventries}
  % \cventry
  %     {Software Team Lead}
  %     {Team Autonomous Underwater Vehicle, IITK}
  %     {Science and Technology Council}
  %     {April 2019 - Present}
  %     {
  %       \begin{cvitems}
  %         \item {Leading a group of 8 people working on the software of Anahita, while managing funding, sponsorships, and technical progress}
  %         \item {Maintaining entire stack of an Autonomous Vehicle, deployed on Git, implemented using ROS, OpenCV and simulation integrated using Gazebo}
  %         \item {Working to participate in the international underwater robotics competition, \textbf{AUVSI RoboSub 2019}, and \textbf{SAUVC 2020}}
  %       \end{cvitems}
  %     }

  %   \cventry
  %     {Secretary}
  %     {IITK Consulting Group}
  %     {Students' Gymkhana} 
  %     {July 2018 - April 2019}
  %     {
  %       \begin{cvitems}
  %         \item {Successfully prepared and delivered lecture to the campus community on introductory Machine Learning and Data Science}
  %         \item {Founding member of the Hobby Group, aiming to work on outsourced consulting projects, with emphasis on insights from collected data}
  %       \end{cvitems}
  %     }

  %   \cventry
  %     {Secretary}
  %     {Robotics Club}
  %     {Students' Gymkhana} 
  %     {April 2018 - April 2019} 
  %     {
  %       \begin{cvitems}
  %         \item {Volunteered in organizing introductory workshops for interested freshman students across the year}
  %         \item{Managed the club website, prepared content for lectures and helped organising competitions for campus community}
  %         \end{cvitems}
  %     }

  %   \cventry
  %     {Academic Mentor and Student Guide}
  %     {Counselling Service}
  %     {IIT Kanpur} 
  %     {April 2018 - April 2019}
  %     {
  %       \begin{cvitems} % Description(s)
  %         \item {Assisted five freshmen students in adjusting to the college environment, providing guidance and emotional support}
  %         \item{Took campus level remedial classes for freshman year Mathematics, provided personal tutoring to academically weak students for their courses}
  %       \end{cvitems}
  %     }

\cvsection{Miscellaneous}

\begin{itemize}
  % \item Developed a Python Application using \textbf{Pygame} 
  %   \ifdefined \ONEPAGE . \else
  %   for 2 player as well as
  %   single player Reversi gameplay as part of ACA Semester Project.
  %   Link -
  %   \href{https://github.com/yashsriv/Reversi-Python}{github://yashsriv/Reversi-Python}
  %   \fi
  %   \vspace{-1mm}
  % \item Ported the educational OS, nachos, to golang. Link -
  %   \href{https://github.com/yashsriv/go-nachos}{github:yashsriv/go-nachos}
  %   \vspace{-1mm}
  % \item Developed an AI for complete-knowledge two-player games in Haskell as a
  %   course project.
  %   \ifdefined \ONEPAGE \else
  %   Implemented Connect 4 with GUI as an instance of that AI.
  %   Link - \href{https://github.com/yashsriv/haskell-connect-4}{github://yashsriv/haskell-connect-4}
  %   \fi
  %   \vspace{-1mm}
  \item Expoited and patched the zoobar server as part of Computer Systems
    Security Course
  \item Developed an android app which was a Websocket Client for a
    Websocket Server hosting a multiplayer game
  \item Contribute to Open Source projects like pdf.js and thelounge
  \item Won Fresher's Science Quiz in inter-hall annual competition
  \item Among the top 15 teams of India in CSAW 2016, CTF
  \item Mentored 6 students in building a chat application
    \ifdefined \ONEPAGE \else
    using nodejs and
    websockets as an Semester Project
    \fi
    \vspace{-1mm}
\end{itemize}
% \cvsection{Interests}

{\fontsize{11pt}{1em}\bodyfontlight\upshape\color{text}
  \begin{itemize}
  \item Open Source
  \item Capture The Flag Contests
  \item Web Development
  \item Image Processing
  \item Artificial Intelligence
  \item Robotics
  \end{itemize}
}

%%% Local Variables:
%%% mode: latex
%%% TeX-engine: xetex
%%% TeX-master: "../cv"
%%% End:


\end{document}

%%% Local Variables:
%%% mode: latex
%%% TeX-engine: xetex
%%% End: