%!TEX TS-program = xelatex
%!TEX encoding = UTF-8 Unicode
% Awesome CV LaTeX Template for CV/Resume
%
% This template has been downloaded from:
% https://github.com/posquit0/Awesome-CV
%
% Author:
% Claud D. Park <posquit0.bj@gmail.com>
% http://www.posquit0.com
%
% Template license:
% CC BY-SA 4.0 (https://creativecommons.org/licenses/by-sa/4.0/)
%


\documentclass[10pt, a4paper]{awesome-cv}
\geometry{left=1.4cm, top=.8cm, right=1.4cm, bottom=1.8cm, footskip=.5cm}
\fontdir[fonts/]

% Color for highlights
% Awesome Colors: awesome-emerald, awesome-skyblue, awesome-red, awesome-pink, awesome-orange
%                 awesome-nephritis, awesome-concrete, awesome-darknight
\colorlet{awesome}{awesome-darknight}
% Uncomment if you would like to specify your own color
% \definecolor{awesome}{HTML}{CA63A8}

% Colors for text
% Uncomment if you would like to specify your own color
% \definecolor{darktext}{HTML}{414141}
% \definecolor{text}{HTML}{333333}
% \definecolor{graytext}{HTML}{5D5D5D}
% \definecolor{lighttext}{HTML}{999999}

% Set false if you don't want to highlight section with awesome color
\setbool{acvSectionColorHighlight}{false}

% If you would like to change the social information separator from a pipe (|) to something else
\renewcommand{\acvHeaderSocialSep}{\quad\textbar\quad}


% Available options: circle|rectangle,edge/noedge,left/right
% \photo{./profile.png}
\name{Yash}{Srivastav}
\position{Senior Undergraduate{\enskip\cdotp\enskip}Computer Science and Engineering}
\address{Indian Institute of Technology, Kanpur}
\mobile{(+91) 705-413-3662}
\email{yash111998@gmail.com}
\homepage{yashsriv.org}
\github{yashsriv}
\linkedin{yashsriv}
% \twitter{@therealyashsriv}
% \quote{``There is no fate but what we make."}

\newcommand{\smallcventry}[6]{\cventry{#1}{#2}{#3}{#4}{#6}}
\newcommand{\specialcvsection}[1]{\cvsection{#1}}




\begin{document}
\makecvheader
\makecvfooter
  {}
  {}
  {\thepage}

\specialcvsection{Educational Qualifications}

\newcommand{\education}[4]{
  & #1 & #2 & &#3 & #4
}
\begin{center}
\begin{tabular}{ | L{0.05cm} l | L{3cm} | L{0.05cm} C{7cm} | r |}
  \hline
  \education{\textbf{Year}}{\textbf{Degree}}{\textbf{Institution(Board)}}{\textbf{CGPA/\%}}\\
  \hline
  \education{July'15 -- June'19 (expected)}{B.Tech, CSE}{Indian Institute of Technology, Kanpur}{9.1/10.0}\\
  \education{2015}{AISSCE -- XII}{Birla High School, Kolkata (CBSE)}{96.6\%}\\
  \education{2013}{ICSE -- X}{AG Church School, Kolkata (CISCE)}{96.6\%}\\
  \hline
\end{tabular}
\end{center}
\vspace{-4mm}

%%% Local Variables:
%%% mode: latex
%%% TeX-master: "../cv.tex"
%%% TeX-engine: xelatex
%%% End:
\cvsection{Honors and Achievements}
\begin{cvhonors}

  \cvhonor
    {2nd in 15+ teams}
    {Student AUV Competition (SAVe), \break organised by\textbf{ NIOT, Chennai} in 2019} 
    {Chennai} 
    {2019}
 
  \cvhonor
  {Top 0.7\%}
  {JEE Advanced (amongst 160,000 candidates)}
  {}
  {2017}

  \cvhonor
  {Top 0.001\%}
  {JEE Main (amongst 1.3 million candidates)}
  {}
  {2017}
  
  \cvhonor
  {Top 1\% (U.P)}
  {National Standard Examination in \textbf{Physics}, 2016, appeared for \textvf{INPho} 2017}
  {India}
  {2016}
 
  \cvhonor
  {Top 1\% (India)} 
  {National Standard Examination in \textbf{Chemistry}, 2016, appeared for \textvf{INCho} 2017}
  {India}
  {}

\end{cvhonors}

%%% Local Variables:
%%% mode: latex
%%% TeX-engine: xetex
%%% TeX-master: "../cv"
%%% End:
\cvsection{Experience}

\begin{cventries}

  \cventry
    {Software Team Member}
    {Team Autonomous Underwater Vehicle, IITK}
    {Prof. Mangal Kothari}
    {May 2018 - Present}
    {
      \begin{cvitems}
        \item {Implemented a novel preprocessing algorithm to formulate a robust underwater vision pipeline, for object detection in a modular fashion}
        \item {Coded parallel nodes running image processing algorithms using OpenCV on the ROS Framework}
        \item {Handled state-of-the art hardware, while preparing to participate in underwater robotics competitions, \textbf{NIOT-SAVe} and \textbf{Singapore SAUVC} }
      \end{cvitems}
    }

  \cventry
    {Backend Software Intern}
    {New York Office, IIT Kanpur}
    {Supervisor: Prof. Manindra Agrawal}
    {May 2018 - July 2018}
    {
      \begin{cvitems}
        \item{Worked on Scala with Akka-HTTP for scalable and concurrent multi threading using functional programming}
        \item{Documented and compiled the entire collection of backend Application Programming Interfaces using PostMan}
        \item{Fixed bugs in the Scala backend, and collaborated using Phabricator, while developing an upcoming social platform}
      \end{cvitems}
    }

  \cventry
    {ESC101 Advanced Track Course Project}
    {Chat-IITK}
    {Mentored by: Prof. Puroshottam Kar}
    {Spring 2018}
    {
      \begin{cvitems}
        \item {Designed and developed a chat application on NodeJS, Express, Socket-IO, and MongoDB}
        \item {Implemented real-time chat using Socket-IO with PassportJS for extensively implemented \textbf{authentication} and \textbf{cookie handling}}
        \item {\textbf{Database management} implemented using MongoDB, and application deployed on Heroku's server}
      \end{cvitems}
    }

\end{cventries}

\cvsection{Skills}
\ifdefined\ONEPAGE
\\
\textbf{Proficient}: C, Golang, Python, Javascript\\
\textbf{Experienced}: C++, Java, Scala, Android\\
\textbf{Exposure}: Haskell, Rust, Dart, Perl\\
\textbf{Web}: Angular, Akka, TypeScript, Redux, Flutter\\
\textbf{Utilities}: Shell Utilities, Git, Docker, Ansible, PostgreSQL, MongoDB, OpenCV,
\LaTeX, Vim, Emacs, Vagrant

\else
\begin{cvskills}

  \cvskill
  {Proficient}
  {C, Golang, Python, Javascript}

  \cvskill
  {Experienced}
  {C++, Java, Scala, Android}
  
  \cvskill
  {Exposure}
  {Haskell, Rust, Dart, Perl}
  
  \cvskill
  {Frameworks}
  {Express.js with Node.js, Akka with Scala, JavaScript, TypeScript, Angular,
    Redux, Flutter}

  \cvskill
  {Utilities}
  {Linux shell utilities, Git, Docker, Ansible, Postgres,
    MongoDB, OpenCV, \LaTeX, Vim, Emacs, vagrant}

\end{cvskills}
\fi
%%% Local Variables:
%%% mode: latex
%%% End:
\cvsection{Relevant Coursework}

\begin{tabular*}{\textwidth}{l l l l}
  Introduction to Programming (A$*$) & Probability \& Statistics  & Introduction to Robotics ($i$) \\
  Data Structures and Algorithm ($i$) & Introduction to Mechanical Design (A$*$)& Introduction to Microeconomics (A) \\
\end{tabular*}
{\footnotesize
    {A$*$: Grade for exceptional performance, $i$: In progress, A: grade}
}
%%% Local Variables:
%%% mode: latex
%%% TeX-engine: xetex
%%% TeX-master: "../cv"
%%% End:
\newpage
\cvsection{Projects}

\begin{cventries}

 \cventry
    {Faculty Advisor: Prof. Mangal Kothari}
    {Team AUV-IITK}
    {Software Team Member}
    {May 2018 - Present}
    {
      \begin{cvitems}
        \item{Designed a \textbf{hierarichal finite state machine} for robust autonomous behavior of the vehicle with failsafes}
        \item{Fused sensor readings from Doppler Velocity Log (DVL) and IMU using an \textbf{EKF} to estimate odometry}
        %\item{Implemented a novel image preprocessing algorithm based on Fusion Framework to formulate a robust underwater computer vision pipeline}
        
        \item{Developed and tested acoustic localization system capable of estimating the Direction of Arrival of ultrasonic underwater signals from pinger, using \textbf{STFT} and \textbf{Cross-Correlation}}
        % \item{Used signal processing operations such as Short Time Fourier Transform and Cross-Correlation to find time delay of arrival between signals}
        % \item{Managed a multi-layered software stack for an autonomous underwater vehicle, Anahita developed on ROS and simulated using Gazebo}
        \item{Tuned and tested Cascaded PID Controller on the vehicle, enabling it to perform waypoint navigation \& visual servoing}
        \item{Extensively used \textbf{Gazebo, a physics engine} to simulate vehicle model in a hydrodynamically realistic environment}
        % \item{Created setups for disparity map generation using a pair of cameras and implemented a modified Fast-SLAM for underwater localization}
      \end{cvitems}
    }
  
   \cventry
    {Mentor: Prof. Mangal Kothari}
    {Realtime Onboard Dense RGB-D Mapping on UAVs}
    {}
    {May 2019 - Present}
    {
      \begin{cvitems}
        \item {Studied and experimented various techniques related to 3D mapping of environment using monocular and stereo cameras on Jetson TX2 for onboard implementation}
        \item {Evaluated approaches for shortcomings and processing requirements while focussing on the scarce size, computation and energy resources on Unmanned Aerial Vehicles (UAVs)}
      \end{cvitems}
    }

  \cventry
    {Mentor: Prof. Shantanu Bhattacharya}
    {Mechanical Quadruped}
    {Course Project -TA202}
    {4th Semester}
    {
      \begin{cvitems}
        \item Designed and simulated a four-legged assembly that uses Jansen's linkage mechanism to walk using \textbf{Solidworks}
        \item Made a working model of the same under constraints of size and materials using manufacturing processes such as lathing, milling and drilling
      \end{cvitems}
    } 

  \cventry
    {Mentor: Prof. Puroshottam Kar}
    {Chat-IITK}
    {Advanced Track Project - ESC101}
    {2nd Semester}
    {
      \begin{cvitems}
        \item {Designed and developed a chat application on NodeJS, Express, and MongoDB, selected in \textbf{12} out of 400+ students}
        \item {Implemented real-time chat using Socket-IO with PassportJS for extensively implemented \textbf{authentication} and \textbf{cookie handling} for session management}
        \item {\textbf{Database management} implemented using MongoDB, and application deployed online on Heroku's server}
      \end{cvitems}
    }

  \smallcventry
    {Robotics Club}
    {Team Humanoid, IITK}
    {IIT Kanpur}
    {Dec 2017 - April 2018} 
    {Software Team Member}
    {
      \begin{cvitems} 
        \item {Worked on a Bipedal Prototype of the humanoid bot, capable of performing statically stable walking}
        \item {Implemented the MATLAB simulated \textbf{inverse kinematics walking algorithm} based on ZMP criteria on the actual robot using a Robot Operating System framework}
        \item {Developed a Web Graphical User Interface for monitoring current status and easier debugging of servos using ROS Web Bridge Server and JavaScript, with a CSS frontend}

      \end{cvitems}
    }


\end{cventries}
\vspace{-2mm}

%%% Local Variables:
%%% mode: latex
%%% TeX-master: "../cv.tex"
%%% TeX-engine: xelatex
%%% End:


% \smallcventry
  % {Self Project}
  % {\href{https://github.com/yashsriv/go-nachos}{go-nachos}}
  % {Operating Systems}
  % {Dec'2017}
  % {}
  % {A port of the educational OS, nachos, in golang}

  % \smallcventry
  % {Course Project, Compiler Design}
  % {\href{https://github.com/yashsriv/tango}{tango} \strong{(\emph{golang to x86 assembly})} }
  % {}
  % {Jan'2018-April'2018}
  % {\emph{\texttt{\href{https://github.com/yashsriv/tango}{github://yashsriv/tango}}}}
  % {
  %   \begin{cvitems}
  %   \item A compiler for go written in go in a team of 3. Compiles from golang
  %     to x86 assembly.
  %   \item Supports a subset of the go language including nested pointers, type
  %     checking, recursion, nested arrays, structs, methods and other common
  %     programming language features.
  %   \item Added a new for comprehension syntax as well to golang.
  %   \end{cvitems}
  % }

  % \smallcventry
  % {Course Project, Computer Architecture}
  % {\href{https://github.com/yashsriv/branch-predictor/blob/master/report/main.pdf}{Branch Predictor}}
  % {Best Predictor}
  % {April'2018}
  % {\emph{\texttt{\href{https://github.com/yashsriv/branch-predictor/blob/master/report/main.pdf}{github://yashsriv/branch-predictor}}}}
  % {
  %   \begin{cvitems}
  %   \item Designed a branch predictor for an intra-class branch prediction
  %     championship based on the CBP-1 framework in a team of 2.
  %   \item Created a modified GEHL predictor with an additional loop predictor.
  %   \item Was adjudged the \textbf{best predictor} amongst all submitted.
  %   \end{cvitems}
  % }

  % \smallcventry
  % {Course Project}
  % {\href{https://github.com/yashsriv/haskell-connect-4}{Connect 4 AI in haskell}}
  % {Functional Programming}
  % {Jan'2018-April'2018}
  % {\emph{\texttt{\href{https://github.com/yashsriv/haskell-connect-4}{github://yashsriv/haskell-connect-4}}}}
  % {
  %   \begin{cvitems}
  %   \item A GUI based connect 4 AI in haskell.
  %   \item Had support for various difficulties and the AI was abstracted out in
  %     order to be able to support any complete knowledge two player game.
  %   \end{cvitems}
  % }

  % \smallcventry
  % {24 Hour Hackathon}
  % {Code.Fun.Do}
  % {Microsoft India}
  % {Sept'2015}
  % {Best 5 ideas}
  % {
  %   \begin{cvitems}
  %   \item Developed an App to help connect teachers and learners based on their
  %     preference of subjects.
  %   \item Used cross-platform \textbf{Universal App Platform} for Windows 10
  %     and a server written in C\#.
  %   \item Was selected as one of the \textbf{best five ideas}.
  %   \end{cvitems}
  % }

  % \smallcventry
  % {Self Project}
  % {\href{https://github.com/yashsriv/go-nachos}{go-nachos}}
  % {Ported nachOS to golang}
  % {Dec'2017}
  % {\emph{\texttt{\href{https://github.com/yashsriv/go-nachos}{github://yashsriv/go-nachos}}}}
  % {}
\cvsection{Positions of Responsibility}

 \begin{cventries}
 \cventry
    {Software Team Lead}
    {Team AUV-IITK}
    {Science and Technology Council}
    {April 2019 - Present}
    {
       \begin{cvitems}
       \item  {Spearheading a group of 8 people working on the software of Anahita, planning and implementing technical changes}
        \item {Maintaining software stack of Autonomous Vehicle, deployed on Git, developed using ROS, OpenCV and Gazebo}
        % \item {Preparing to participate in the international underwater robotics competition, \textbf{AUVSI RoboSub}, and \textbf{SAUVC}}
       \end{cvitems}
    }
  \end{cventries}

\begin{itemize}
\item \textbf{Secretary}, \emph{Robotics Club, IIT Kanpur 2018-19}
\item \textbf{Secretary}, \emph{Consulting Hobby Group, IIT Kanpur 2018-19}
\item \textbf{Student Guide}, \emph{Counselling Service, 2018-19}
\item \textbf{Academic Mentor}, \emph{Counselling Service, 2018-19}
\end{itemize}
  %   \cventry
  %     {Secretary}
  %     {IITK Consulting Group}
  %     {Students' Gymkhana} 
  %     {July 2018 - April 2019}
  %     {
  %       \begin{cvitems}
  %         \item {Successfully prepared and delivered lecture to the campus community on introductory Machine Learning and Data Science}
  %         \item {Founding member of the Hobby Group, aiming to work on outsourced consulting projects, with emphasis on insights from collected data}
  %       \end{cvitems}
  %     }

  %   \cventry
  %     {Secretary}
  %     {Robotics Club}
  %     {Students' Gymkhana} 
  %     {April 2018 - April 2019} 
  %     {
  %       \begin{cvitems}
  %         \item {Volunteered in organizing introductory workshops for interested freshman students across the year}
  %         \item{Managed the club website, prepared content for lectures and helped organising competitions for campus community}
  %         \end{cvitems}
  %     }

  %   \cventry
  %     {Academic Mentor and Student Guide}
  %     {Counselling Service}
  %     {IIT Kanpur} 
  %     {April 2018 - April 2019}
  %     {
  %       \begin{cvitems} % Description(s)
  %         \item {Assisted five freshmen students in adjusting to the college environment, providing guidance and emotional support}
  %         \item{Took campus level remedial classes for freshman year Mathematics, provided personal tutoring to academically weak students for their courses}
  %       \end{cvitems}
  %     }

\cvsection{Miscellaneous}

\begin{itemize}
  % \item Developed a Python Application using \textbf{Pygame} 
  %   \ifdefined \ONEPAGE . \else
  %   for 2 player as well as
  %   single player Reversi gameplay as part of ACA Semester Project.
  %   Link -
  %   \href{https://github.com/yashsriv/Reversi-Python}{github://yashsriv/Reversi-Python}
  %   \fi
  %   \vspace{-1mm}
  % \item Ported the educational OS, nachos, to golang. Link -
  %   \href{https://github.com/yashsriv/go-nachos}{github:yashsriv/go-nachos}
  %   \vspace{-1mm}
  % \item Developed an AI for complete-knowledge two-player games in Haskell as a
  %   course project.
  %   \ifdefined \ONEPAGE \else
  %   Implemented Connect 4 with GUI as an instance of that AI.
  %   Link - \href{https://github.com/yashsriv/haskell-connect-4}{github://yashsriv/haskell-connect-4}
  %   \fi
  %   \vspace{-1mm}
  \item Expoited and patched the zoobar server as part of Computer Systems
    Security Course
  \item Developed an android app which was a Websocket Client for a
    Websocket Server hosting a multiplayer game
  \item Contribute to Open Source projects like pdf.js and thelounge
  \item Won Fresher's Science Quiz in inter-hall annual competition
  \item Among the top 15 teams of India in CSAW 2016, CTF
  \item Mentored 6 students in building a chat application
    \ifdefined \ONEPAGE \else
    using nodejs and
    websockets as an Semester Project
    \fi
    \vspace{-1mm}
\end{itemize}
% \cvsection{Interests}

{\fontsize{11pt}{1em}\bodyfontlight\upshape\color{text}
  \begin{itemize}
  \item Open Source
  \item Capture The Flag Contests
  \item Web Development
  \item Image Processing
  \item Artificial Intelligence
  \item Robotics
  \end{itemize}
}

%%% Local Variables:
%%% mode: latex
%%% TeX-engine: xetex
%%% TeX-master: "../cv"
%%% End:


\end{document}

%%% Local Variables:
%%% mode: latex
%%% TeX-engine: xetex
%%% End: